%!TEX root = ../template.tex
%%%%%%%%%%%%%%%%%%%%%%%%%%%%%%%%%%%%%%%%%%%%%%%%%%%%%%%%%%%%%%%%%%%%
%% chapter4.tex
%% NOVA thesis document file
%%
%% Chapter with lots of dummy text
%%%%%%%%%%%%%%%%%%%%%%%%%%%%%%%%%%%%%%%%%%%%%%%%%%%%%%%%%%%%%%%%%%%%
\chapter{Requerimientos}
\label{cha:Requerimientos}

El presente trabajo de investigación se consideró con los siguientes requerimientos, un estudio descriptivo debido a la comprensión de aspectos cualitativos y cuánticos, su propósito es el planteamiento y descripción de un problema y propone estrategias en el diseño de red SD-WAN.

\section{Requerimientos Funcionales}
\label{sec:Requerimientos Funcionales}

La aplicación de estratégias en diseño de red SD-WAN contribuye con calidad de servicio en la protección de datos, fácil administración. La investigación sugiere preguntas que a continuación se relacionó con los requerimientos funcionales de partida.

\begin{itemize}
\item[•]\textbf{Posibilidades:} etapa de conceptualización del proceso de diseño.
\item[•]\textbf{Efectividad:} modelamiento del sistema puede ser evaluado por novedad y viabilidad.
\item[•]\textbf{Estructura:} ordenador o herramientas necesarias para el diseño e implementación.
\item[•]\textbf{Método:} solución del problema.
\end{itemize}
El avance de la tecnología ha creado nuevos sistemas y herramientas para los sistemas de telecomunicaciones.

\section{Requerimientos no Funcionales}
\label{sec:Requerimientos no Funcionales}
La implementación de políticas del diseño de una red SD-WAN empresa reduce los riesgos, protección de datos, mantenimiento, fácil administración, entre otros.

Contribución a la calidad de servicio (QOS) y navegación segura. Personal calificado para capacitación, crea conciencia de seguridad en los empleados de la organización.
