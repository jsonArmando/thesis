%!TEX root = ../template.tex
%%%%%%%%%%%%%%%%%%%%%%%%%%%%%%%%%%%%%%%%%%%%%%%%%%%%%%%%%%%%%%%%%%%%
%% chapter4.tex
%% NOVA thesis document file
%%
%% Chapter with lots of dummy text
%%%%%%%%%%%%%%%%%%%%%%%%%%%%%%%%%%%%%%%%%%%%%%%%%%%%%%%%%%%%%%%%%%%%
\chapter{Metodología de Desarrollo}
\label{cha:Metodología de Desarrollo}

\textit{Para el desarrollo del presente proyecto de grado se tomó en cuenta: “El método de investigación cualitativa es la recogida de información basada en la observación de comportamientos naturales, discursos, respuestas abiertas para la posterior interpretación de significados. Investigadores cualitativos estudian la realidad en su contexto natural.”}

\section{Utilización de la Metodología del Project Management Institute (PMI)}
\label{sec:Utilización de la Metodología del Project Management Institute (PMI)}

\textit{Método PMI “Ofrece una serie de directrices que orientan la gestión y dirección de proyectos, válidas para la gran mayoría de proyectos. Sin embargo, este método no debe concebirse como algo cerrado.”}.

Tomando en cuenta los siguientes procesos de desarrollo del presente trabajo de grado.

\textit{“Un proceso está compuesto por todas aquellas actividades interrelacionadas que se deben ejecutar para poder obtener el producto o prestar el servicio. Existen dos tipos de procesos que se superponen e interactúan entre sí.}

Procesos de la dirección de proyectos. Compuesto por cinco procesos o categorías diferentes, estos procesos, aseguran el progreso adecuado del proyecto a lo largo de todo su ciclo de vida.

\begin{itemize}
\item[•] \textbf{Ciclo de vida.}
\item[•] \textbf{Proceso de iniciación.}
\item[•] \textbf{Proceso de planificación.}
\item[•] \textbf{Proceso de ejecución.}
\item[•] \textbf{Proceso de supervisión y control.}
\item[•] \textbf{Proceso de cierre del proyecto.}
\item[•] \textbf{Procesos orientados al producto. Este tipo de procesos especifican y crean el producto. Varían en función del área de conocimiento.}
Con las siguientes áreas de conocimiento:

\item[•] \textbf{Gestión de la Integración.}
\item[•] \textbf{Gestión del Alcance.}
\item[•] \textbf{Gestión del Tiempo.}
\item[•] \textbf{Gestión de Costes.}
\item[•] \textbf{Gestión de la Calidad.}
\item[•] \textbf{Gestión de los Recursos Humanos.}
\end{itemize}
