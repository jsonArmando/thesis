%!TEX root = ../template.tex
%%%%%%%%%%%%%%%%%%%%%%%%%%%%%%%%%%%%%%%%%%%%%%%%%%%%%%%%%%%%%%%%%%%
%% chapter1.tex
%% NOVA thesis document file
%%
%% Chapter with introduciton
%%%%%%%%%%%%%%%%%%%%%%%%%%%%%%%%%%%%%%%%%%%%%%%%%%%%%%%%%%%%%%%%%%%
\newcommand{\novathesis}{\emph{novathesis}}
\newcommand{\novathesisclass}{\texttt{novathesis.cls}}


\chapter{Introducción}
\label{cha:Introducción}

\begin{quotation}
  \itshape
   Entender a un ingeniero es más complicado que una \textcolor{blue}{Transformada Rápida de Fourier},  sus pensamiento son como un circuito integrado.  
\end{quotation}

\section{Título} % (fold)
\label{sec:Título}

\textbf{DISEÑO Y SIMULACIÓN PARA CLIENTE SD-WAN PARA CLIENTE
RETAIL CON CONSUMO DE SERVICIOS EN LA NUBE.}
\section{Presentación} % (fold)
\label{sec:Presentación}

The \novathesis\ was originally developed to help MSc and PhD students of the Computer Science and Engineering Department of the Faculty of Sciences and Technology of NOVA University of Lisbon (DI-FCT-NOVA) to write their thesis and dissertations Using \LaTeX.
%
These student can easily cope with \LaTeX\ by themselves, and the only need some help in the bootstrap process to make their life easier.

However, as the template spread out among the students from other degrees at FCT-NOVA, the demand for am easier-to-use template as grown.
%
And the template in its current shape aims at answering the expectations of those that, although they are not familiar with programming nor with markup languages, so still feel brave enough to give \LaTeX\ a try and rejoice with the beauty of the texts typeset by this system.

% section a_bit_of_history (end)


\section{Definición del Problema} % (fold)
\label{sec:Definición del Problema}

Un cliente del sector retail como parte de su proyecto de renovación tecnológica se encuentra migrando sus servicios y aplicaciones internas a la nube, el cliente es consciente de que esta migración generaría mucha mayor carga sobre sus enlaces WAN, y se consideran inviables las ampliaciones de todos sus canales principal y Backup para el tráfico estimado ya que esto aumentaría los costos de tal forma que se haría inviable. Además de la migración a la nube este es un cliente que se encuentra creciendo a un ritmo muy acelerado y cuenta en el momento con alrededor de 700 sedes remotas, por lo cual con la infraestructura actual a veces no es capaz de darle manejo a todo el tráfico que tiene cuando se presentan picos.
\\
\\
La gestión de la red se realiza de forma manual en cada equipo, y al contar con tantas sedes los cambios y la implementación de las políticas de red se ejecutan de forma muy lenta y por tanto realizar cambios a nivel de IT se vuelve muy complicado dado el cuello de botella en la gestión de la red, lo cual aumenta los tiempos de ejecución de los cambios de red para el cliente.
El cliente presenta un aumento de tráfico que supera la capacidad de sus enlaces WAN actuales al migrar sus servicios a la nube, dicho aumento afecta la calidad de los servicios en tiempo real como la telefonía y los servicios de videoconferencia. Al validar los costos de las ampliaciones necesarias para soportar la cantidad de tráfico se identifica que el costo recurrente mensual es excesivo para el presupuesto de la compañía por lo que se debe encontrar una alternativa que se ajuste tanto a las necesidades como al presupuesto del cliente. Adicionalmente cuando se presentan fallas en la MPLS el cliente debe conmutar su tráfico al datacenter de forma manual, lo cual aumenta los tiempos de gestión de las fallas y por tanto la indisponibilidad del servicio. Adicional a estos problemas de disponibilidad y de saturación se han presentado ataques de seguridad sobre la infraestructura del cliente y robo de información utilizando los canales de Internet que tiene el cliente y los datos que por allí transporta.
\\
\\
El cliente es una de las compañías líderes del sector retail en Colombia, con alrededor de 700 sucursales a nivel nacional y con 11 oficinas regionales que se encargan de la administración de estas sucursales,	cada una de las sucursales se encuentra  conectada por túneles L2TP hacia su respectiva regional, estos túneles son formados a través de enlaces de internet banda ancha y mediante ellos se accede a los servicios de red, algunos de estos servicios como telefonía IP, servidor de archivos y directorio activo se encuentran ubicados en el centro de datos privado del cliente, mientras que otros servicios como SAP y la interconexión con instituciones financieras y con sus aliados estratégicos se encuentran como servicios virtualizados en grandes centros de datos. Adicional a esto el cliente se encuentra utilizando servicios en la nube como skype para colaboración, Gsuite y algunos servicios de Amazon.
\\
\\
La comunicación con cada uno de estos servicios se establece desde Internet para el caso de las sucursales, para el caso de sus regionales y el centro de datos en donde se encuentran sus servicios virtualizados esta comunicación se establece mediante los canales MPLS presentes en cada una de las regionales y el backup de esta comunicación son túneles EoIP mediante el canal de internet de cada regional. A continuación se muestra la topología de la compañía que muestra la forma como se interconectan las sedes regionales entre sí y con el centro de datos desde el cual se accede a los servicios críticos de la compañía:
\\
\\
Se identifican como causas de los inconvenientes anteriormente mencionados la utilización de servicios en la nube y el hecho de que cada una de las sucursales debe enviarle el tráfico a las regionales para consumir cualquier recurso de red, inclusive si es una llamada a otra sucursal esto ha ocasionado los altos picos de tráfico sobre los canales de intranet de las sucursales. Adicional a esto las sucursales cuentan con túneles EoIP configurados entre las sedes en caso de falla de su canal MPLS, pero los túneles son utilizados únicamente como backup, por lo que el ancho de banda de los canales de internet no es utilizado aún cuando se presentan picos de saturación sobre la intranet.
\\
\\ 
Por otro lado la causa de la lentitud en la configuración de nuevas políticas o servicios de red es el hecho de que los cambios se realizan manualmente, es por este motivo que dentro de la solución se propondrá el hecho de que haya gestión centralizada desde la controladora SD-WAN. En cuanto a los problemas de seguridad presentados se incluye dentro de la solución el cifrado de los túneles que interconectan tanto las sucursales como las oficinas regionales, de manera que el tráfico deje de cursar en texto claro por la red pública.
\\
\\
Por otro lado una de las causas más importantes de los problemas de disponibilidad de servicio que ha presentado el cliente ha sido que bajo el modelo actual la conmutación de sus servicios de Datacenter se realiza a través de unas VPN IPSEC que se suben manualmente en los equipos, lo que incrementa el tiempo de indisponibilidad de los servicios y los tiempos de gestión de fallas.
\\
\\
\section{Aspectos a Solucionar} % (fold)
\label{sec:Aspectos a Solucionar}

La gestión de la infraestructura de red debe realizarse de forma manual en cada una de las tiendas.
\\
	La comunicación por internet entre las diferentes regionales se realiza sin cifrar y la de las tiendas se cifra bajo un protocolo que ya no es considerado seguro.
\\
	La conexión hacia el centro de datos no cuenta con un respaldo automático sino que en este momento debe realizarse de forma manual lo cual aumenta el tiempo de gestión de una falla y por lo tanto disminuye el tiempo de disponibilidad.
\\
	En momentos de congestión de la red el tráfico cursa únicamente por el canal principal de la MPLS y el ancho de banda disponible por el canal de internet no es aprovechado.
\\
La conexión de las tiendas hacia todos los servicios que consume depende del canal de internet de la regional, si este se cae todas las tiendas que están asociadas a él quedan sin conexión.

\section{Solución Propuesta} % (fold)
\label{sec:Solución Propuesta}

Se propone realizar un diseño para el cambio de esquema de conectividad WAN del cliente de una solución tradicional a una solución SD-WAN que permita realizar los cambios de forma centralizada y más ágil, esta automatización debe realizarse en conjunto con políticas de conectividad que le garanticen al cliente el balanceo de carga del tráfico WAN de manera eficiente e inteligente utilizando los enlaces dependiendo de las necesidades del tráfico de cada aplicación.
\\
\\
	La solución debe incluir además un esquema de transporte que de independencia del medio o servicio que se utilice (Internet o Intranet) y que permita tanta flexibilidad de cambiar el tipo de servicio de manera transparente cómo reducir los costos mensuales del cliente en cuanto a enlaces WAN, esto debe realizarse con el protocolo de enrutamiento que más se ajuste al esquema y con las políticas de QoS necesarias para garantizar que el tráfico de cada servicio funcione de forma adecuada.
\\
\\
	La solución debe diseñarse además de forma que todos los aspectos mencionados anteriormente apliquen tanto para el tráfico que el cliente utilice para aplicaciones en la nube como para el tráfico de aplicaciones que aún se encuentren en Datacenter administrado por ellos o en el centro de datos del ISP.
\\
\\
	El cliente requiere por tanto una solución de SD-WAN que disminuya los costos de la operación y al mismo tiempo incremente la disponibilidad de ancho de banda y eficiencia de sus conexiones WAN mediante un balanceo de carga entre sus enlaces principal y de respaldo. El cliente requiere un diseño de red que cumpla con los siguientes criterios:

\begin{itemize}
\item[•]Balanceo de tráfico inteligente: el cliente requiere que sea utilizado el ancho de banda de los dos canales que tiene en cada sede para soportar la cantidad de tráfico que implica su migración de servicios a la nube, este balanceo debe ser inteligente de manera que se cumpla con los requisitos de retardo, jitter y pérdida de paquetes que requiere cada aplicación de la compañía, si estos criterios no se cumplen bajo uno de los canales el tráfico debe ser enviado por el otro de forma automática.

\item[•]Seguridad: Al tratarse de tráfico transaccional el cliente requiere que el transporte de datos cumpla con todos los requisitos de seguridad en la compañía en cuanto a la Integridad, privacidad y disponibilidad.

\item[•]Disponibilidad: Se requiere que el servicio tenga una alta disponibilidad y que esta se priorice para las aplicaciones críticas del cliente, el esquema de alta disponibilidad debe ser automático.

\item[•]Aprovisionamiento ágil: Se requiere que en caso de requerir cambios generales a nivel de red WAN estos no tengan que ser configurados de forma manual en cada una de las sedes, sino que por el contrario puedan configurarse políticas de forma centralizada y enviarse las configuraciones de forma masiva para agilizar la implementación de cambios.

\item[•]Independencia del transporte: Se requiere una solución que no dependa de la forma de transporte, que pueda establecerse por Internet o por MPLS sin inconvenientes y que si se decide cambiar de tecnología esto sea transparente para el servicio.

\item[•]Adecuado para nube híbrida: La solución propuesta debe cumplir los requerimientos tanto para las aplicaciones que se encuentran en la nube como para aquellas que aún están en el datacenter del cliente, y debe realizar balanceo y dar prioridad a las aplicaciones.

\item[•]Calidad de servicio: El diseño debe tener unas políticas de QoS que garanticen el correcto funcionamiento de todas las aplicaciones que cursan por la red, que incluyen tráfico de voz y video.                                 

\item[•]Conexiones dinámicas: El diseño propuesto debe utilizar tecnologías que eliminen la necesidad de configurar túneles estáticos cada vez que se agregue una sede o regional sino que estos se configuran dinámicamente en una tecnología en malla.
\end{itemize}

\section{Metodología} % (fold)
\label{sec:Metodología}


\section{Contribuciones} % (fold)
\label{sec:Contribuciones}

\section{Estructura de la Tesis} % (fold)
\label{sec:Estructura de la Tesis}