Several companies in different fields of work today are implementing SD-WAN \textbf{("Software Defined Networking in Wide Area Networks")}, to counteract the high cost of telecommunications equipment and better management of their infrastructures. To finance their projects have searched for alternatives such as cloud or cloud, virtualization among others. However, companies have not sought migration or adaptation of their networks to this new technology, creating a gap in underdevelopment in this field of "Networking".
\\
\\
The objective of the study is to determine how the virtualization of new networks can be increased and improve in the companies that need it. To this end, the research question is as follows: Is it necessary to implement SD-WAN \textbf {("Software Defined Networking in Wide Area Networks")}? In this context, it is a technology that is very necessary today.
\\
\\
The studies carried out suggest that an SD-WAN \textbf {("Software Defined Networking in Wide Area Networks")} must be implemented and designed for a Retail Client, in order to improve its functionality in the networks, obtain better administration and risk to vulnerabilities describing several qualitative and quantitative strategies to give a solution that converges to improve the infrastructure of the company. 

% Palavras-chave do resumo em Português
\begin{keywords}
\textbf {SD-WAN, MPLS, Virtualization, Cloud, Capex, Opex, Networks} \ldots
\end{keywords}
