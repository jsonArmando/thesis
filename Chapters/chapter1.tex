%!TEX root = ../template.tex
%%%%%%%%%%%%%%%%%%%%%%%%%%%%%%%%%%%%%%%%%%%%%%%%%%%%%%%%%%%%%%%%%%%%
%% chapter4.tex
%% NOVA thesis document file
%%
%% Chapter with lots of dummy text
%%%%%%%%%%%%%%%%%%%%%%%%%%%%%%%%%%%%%%%%%%%%%%%%%%%%%%%%%%%%%%%%%%%%
\chapter{Justificación}
\label{cha:Justificación}


La administración centralizada de la red mediante una solución SD-WAN permitiría realizar cambios de políticas de red de forma mucho más ágil, evitando que la red se vuelva un cuello de botella para la ejecución de proyectos de TI.
\\
\\
	La solución propuesta mejoraría el comportamiento de la red en varios aspectos, principalmente en disponibilidad y calidad de servicio pero también en otras áreas críticas para la organización como la seguridad, es un rediseño completo que permitiría a la empresa ser más competitiva con unos mayores tiempos de disponibilidad de sus servicios tanto para las regionales como para las tiendas y con una menor carga de trabajo sobre la gestión de la red al asegurarse de que la conmutación de los servicios se realice de forma automática y al garantizar una utilización más eficiente de los recursos de la red al realizar el balanceo de carga.
\\
\\
	El esquema de balanceo de carga asegura que los recursos de la red se utilicen de forma más eficiente y que de esta manera cuando el cliente tenga picos de tráfico no se vea afectado todo el tráfico por un enlace mientras que el otro enlace se encuentra disponible y podría utilizarse, además de esto el balanceo reduce costos ya que aumenta la capacidad real de la conexión WAN y por tanto no se requerirían ampliaciones de ancho de banda por el momento.
\\
\\
	La disponibilidad también mejoraría notablemente al eliminar la dependencia de la conexión de las tiendas con el canal de Internet de una sola regional, esta dependencia se eliminaría bajo un esquema de túneles dinámicos que cambian el actual comportamiento de topología estrella a una topología en malla, dichos túneles dinámicos contarían además con los niveles de encripción adecuados para mitigar los problemas de seguridad que se han presentado.
\\
\\
	Un cambio de esta magnitud tomaría mucho tiempo bajo el modelo de gestión actual, si en cambio se utiliza una red SD-WAN con gestión centralizada de los recursos de red el cambio se haría de forma mucho más ágil, lo que al final representa menores costos para la compañía.
