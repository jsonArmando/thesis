%!TEX root = ../template.tex
%%%%%%%%%%%%%%%%%%%%%%%%%%%%%%%%%%%%%%%%%%%%%%%%%%%%%%%%%%%%%%%%%%%%
%% chapter4.tex
%% NOVA thesis document file
%%
%% Chapter with lots of dummy text
%%%%%%%%%%%%%%%%%%%%%%%%%%%%%%%%%%%%%%%%%%%%%%%%%%%%%%%%%%%%%%%%%%%%
\chapter{Introducción}
\label{cha:Introducción}
%\begin{quotation}
%  \itshape
%   Entender a un ingeniero es más complicado que una \textcolor{blue}{Transformada Rápida de Fourier},  sus %pensamiento son como un circuito integrado. 
%   \\
%   \begin{flushright}
%   \raggedright\textit{Anónimo}
%   \end{flushright}
%\end{quotation}

El presente proyecto corresponde al tema a SD-WAN \textbf{(“Software Defined Networking in Wide Area Networks”)} significa en español solución de conectividad definida por software para redes de área extensa. Es una herramienta que se usa exclusivamente en el campo de las telecomunicaciones dispuesto en una red o sistema para ser un dispositivo de hardware de virtualización que ejecuta su propio procedimiento sobre circuitos, realizando funciones de enrutado, con el que los administradores pueden desplegarse en un costo reducido en gran cantidad de nodos de la red.
\\
\\
La característica principal de esta tecnología de la información se ocupa en la protección de datos, simula servicios, programas, aplicaciones redes, posibilidad de personalizar cada dispositivo de forma local y controlar una arquitectura de telecomunicaciones de forma centralizada.
\\
\\
Para analizar los SD-WAN \textbf{(“Software Defined Networking in Wide Area Networks”)} es necesario mencionar sus acciones, conectividad en el entorno de negocio empresarial,  debido un administrador de red no puede cubrir dichas necesidades solo con servicios de MPLS \textbf{("del inglés Multiprotocol Label Switching")} WAN \textbf{("Wide Area Network en inglés")} para interconectar DataCenter y oficinas remotas.
\\
\\
La investigación de esta tecnología se realizó por el interés de conocer la creación de redes híbridas que adicionan múltiples tecnologías de acceso, incluyendo servicios de Internet, enrutamiento de tráfico dinámico disponiendo en tiempo real la conectividad.
\\
\\
Ante los retos de las redes modernas, en donde el reto consiste en aumentar la disponibilidad, confiabilidad y seguridad de una red mientras se reducen los costos de CAPEX y OPEX y para un cliente que se enfrenta a una transformación digital en donde sus aplicaciones se encontrarán ahora con un esquema de red híbrida en lugar de datacenter tradicional, al migrar muchas de sus aplicaciones claves de negocio a la nube. Se requiere una solución que simplifique la gestión y la administración de una red compleja con cientos de sitios branch y permita aprovechar los beneficios de la nube sin suponer esto un aumento demasiado grande en los costos de OPEX. Ante estos retos SD-WAN \textbf{(“Software Defined Networking in Wide Area Networks”)} se plante hoy día como una de las mejores soluciones para gozar de una red inteligente y simple que esté enfocada al uso de las aplicaciones.
\\
\\
El proyecto como tal plantea el diseño y simulación para un cliente retail en este caso, que cuenta con cientos de redes \textit{"branch"} alrededor de todo el país y requiere mejorar la disponibilidad y confiabilidad de su red al mismo tiempo que la adapta a las tecnologías cloud que ya se encuentran adaptando a su negocio.

