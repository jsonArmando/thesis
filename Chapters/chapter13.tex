%!TEX root = ../template.tex
%%%%%%%%%%%%%%%%%%%%%%%%%%%%%%%%%%%%%%%%%%%%%%%%%%%%%%%%%%%%%%%%%%%%
%% chapter4.tex
%% NOVA thesis document file
%%
%% Chapter with lots of dummy text
%%%%%%%%%%%%%%%%%%%%%%%%%%%%%%%%%%%%%%%%%%%%%%%%%%%%%%%%%%%%%%%%%%%%
\chapter{Conclusiones}
\label{cha:Conclusiones}

La solución cumple con los requerimientos del proyecto según la simulación realizada, ya que se está realizando el balanceo de carga sobre los diferentes enlaces utilizando PfR y la redundancia utilizando EIGRP como protocolo de enrutamiento y HSRP en el caso de la topología de dos routers.
\\
\\
Se establece el diseño para los diferentes escenarios que maneja el cliente, un diseño para el centro de datos, uno para las sedes nacionales, para las sedes principales y finalmente el diseño de las tiendas. Esto permite adaptar la solución a los diferentes escenarios del cliente.
\\
\\
El diseño de calidad de servicio permite establecer la prioridad necesaria para tráfico sensible al retardo, jitter y pérdida de paquetes y por lo tanto mejorar la experiencia del usuario final.
\\
\\
Al realizar el enrutamiento basado en aplicación y establecer umbrales según sus necesidades de cada una de ellas se garantiza una mejora en la experiencia de los usuarios finales en el uso de todas las aplicaciones.
\\
\\
El diseño utilizando una controladora centralizada permite realizar cambios y aprovisionar servicios de forma mucho más rápida y eficiente, generando una disminución en las tareas repetitivas que debe realizar el departamento de IT.
\\
\\
Con MPLS, los proveedores tenían un interés comercial en minimizar la latencia, en parte optimizando su enrutamiento. De lo contrario, los clientes se sentirían molestos, lo que aumentaría su insatisfacción. En última instancia, los proveedores verían una mayor rotación de clientes y pérdidas de ingresos. Pero los proveedores de la red troncal de Internet buscan maximizar el valor de sus redes, no el rendimiento de ninguna aplicación. 
\\
\\
A menudo, puede tener más sentido descargar el tráfico en la red troncal de otro proveedor que trasladarlo por una ruta más rápida a lo largo de su propia red. Así es como terminas con el "enrutamiento" demasiado familiar para los ingenieros de Internet.
\\
\\
Gran parte de los problemas con el enrutamiento de Internet ocurren en el núcleo de la red. Cuando el tráfico se mantiene dentro de la región, el impacto del núcleo de Internet a menudo se minimiza. Una diferencia del 20\% en una ruta de 20 ms es insignificante para la mayoría de las aplicaciones. Pero la misma variación en una ruta de 200 ms puede significar la diferencia entre una llamada de voz clara y una ininteligible.

