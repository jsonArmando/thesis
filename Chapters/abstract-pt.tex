Varias empresas de diferentes campos laborales hoy en día estan implementado SD-WAN \textbf{(“Software Defined Networking in Wide Area Networks”)}, para contrarestar el alto costo de equipos de telecomunicaciones y mejor administración de sus infraestructuras. Para financiar sus proyectos han buscado alternativas como la nube o cloud, virtualización entre otros. Sin embargo actualmente las empresas no han buscado la migración o adaptación de sus redes a esta nueva tecnología, creando una brecha en subdesarrollo en este campo de "Networking". 
\\
\\
El objetivo de estudio es determinar cómo la virtualización de nuevas redes puede aumentarse y mejorar en las empresas que lo necesitan, con este fin la pregunta de investigación es la siguiente, ¿Es necesario la implementación de SD-WAN \textbf{(“Software Defined Networking in Wide Area Networks”)}? En este contexto es una tecnología que hoy en día es muy necesario.
\\
\\
Los estudios realizados sugiere que se debe llevar a cabo una implementación y diseño para un Cliente Retail una SD-WAN \textbf{(“Software Defined Networking in Wide Area Networks”)}, para mejorar su funcionalidad en las redes, obtener mejor administración y riesgo a vulnerabilidades describiendo varias estrategias cualitativas y cuantitativas para dar una solución que converja a mejorar la infraestructura de la empresa

% Palavras-chave do resumo em Português
\begin{keywords}
\textbf{SD-WAN, MPLS, Virtualización, Cloud, Capex, Opex, Redes}\ldots
\end{keywords}
% to add an extra black line
