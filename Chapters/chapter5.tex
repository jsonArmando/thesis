%!TEX root = ../template.tex
%%%%%%%%%%%%%%%%%%%%%%%%%%%%%%%%%%%%%%%%%%%%%%%%%%%%%%%%%%%%%%%%%%%%
%% chapter4.tex
%% NOVA thesis document file
%%
%% Chapter with lots of dummy text
%%%%%%%%%%%%%%%%%%%%%%%%%%%%%%%%%%%%%%%%%%%%%%%%%%%%%%%%%%%%%%%%%%%%
\chapter{Glosario de Términos}
\label{cha:Glosario de Términos}

	\textbf{L2TP:} Layer 2 Tunneling protocol. es un protocolo utilizado por redes privadas virtuales que fue diseñado por un grupo de trabajo de IETF como el heredero aparente de los protocolos PPTP y L2F, creado para corregir las deficiencias de estos protocolos y establecerse como un estándar aprobado por el IETF (RFC 2661).
	\\
	\\
	\textbf{EOIP:} Túnel Ethernet over IP.
	\\
	\\
	\textbf{ISP:} Proveedor de servicios de internet. El proveedor de servicios de Internet, (ISP, por la sigla en inglés de Internet Service provider) es la empresa que brinda conexión a Internet a sus clientes. Un ISP conecta a sus usuarios a Internet a través de diferentes tecnologías como DSL, cablemódem, GSM, dial-up, etcétera.
	\\
	\\
	\textbf{SD-WAN:} Software Defined - Wide Area Network.
	\\
	\\
	\textbf{QoS:} Quality of Service.
	\\
	\\
	\textbf{WAN:} Wide Area Network.
	\\
	\\
	\textbf{TI:} Tecnologías de la información.
	\\
	\\
	\textbf{RETAIL:} Sector economico que engloba a las empresas especializadas en la venta masiva de productos.
	\\
	\\
	\textbf{DATACENTER:} Centro de datos.
	\\
	\\
	\textbf{EGRP:} (Protocolo de Enrutamiento de Puerta de enlace Interior Mejorado en español) es un protocolo de encaminamiento vector distancia avanzado, propiedad de Cisco Systems, que ofrece lo mejor de los algoritmos de vector de distancias y del estado de enlace.
	\\
	\\
	\textbf{DMVPN:} La red privada virtual dinámica multipunto es una forma dinámica de túnel de una red privada virtual compatible con enrutadores basados en Cisco IOS, enrutadores Huawei AR G3 y firewalls USG, y en sistemas operativos tipo Unix.
	\\
	\\
	\textbf{APIC-EM:} Controlador de infraestructura de políticas de aplicaciones (APIC) de Cisco es el punto unificador de automatización y administración para la estructura de Infraestructura centrada en aplicaciones (ACI).
	\\
	\\
	\textbf{MD-SAL:} Model-Driven SAL (MD-SAL) es un conjunto de servicios de infraestructura destinados a brindar soporte común y genérico a los desarrolladores de aplicaciones y complementos.
	\\
	\\	
	\textbf{BGP:} En telecomunicaciones, el protocolo de puerta de enlace de frontera o BGP (del inglés Border Gateway Protocol) es un protocolo mediante el cual se intercambia información de encaminamiento entre sistemas autónomos.
	\\
	\\
	\textbf{LISP:} (Históricamente LISP) es una familia de lenguajes de programación de computadora de tipo multiparadigma con larga historia y una inconfundible y útil sintaxis basada en la notación polaca.
	\\
	\\	
	\textbf{SNMP:}
	El Protocolo simple de administración de red oSNMP (del inglés Simple Network Management Protocol) es un protocolo de la capa de aplicación que facilita el intercambio de información de administración entre dispositivos de red.
	\\
	\\	
	\textbf{NBMA:}
	Una red de acceso múltiple no de difusión es una red informática a la que se conectan múltiples hosts, pero los datos se transmiten solo directamente desde una computadora a otro único host a través de un circuito virtual o a través de un tejido conmutado.
